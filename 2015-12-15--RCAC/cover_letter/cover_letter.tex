

%% start of file `template.tex'.
%% Copyright 2006-2013 Xavier Danaux (xdanaux@gmail.com).
%
% This work may be distributed and/or modified under the
% conditions of the LaTeX Project Public License version 1.3c,
% available at http://www.latex-project.org/lppl/.


\documentclass[11pt,a4paper,sans]{moderncv}     

\moderncvstyle{classic}    % style options are 'casual' (default), 'classic', 'oldstyle' and 'banking'
\moderncvcolor{green}      % color options 'blue' (default), 'orange', 'green', 'red', 'purple', 'grey' and 'black'
%\renewcommand{\familydefault}{\sfdefault}         % to set the default font; use '\sfdefault' for the default sans serif font, '\rmdefault' for the default roman one, or any tex font name
%\nopagenumbers{}

\usepackage{fontspec}

\usepackage[scale=0.79]{geometry}
%\setlength{\hintscolumnwidth}{3cm}                % if you want to change the width of the column with the dates
%\setlength{\makecvtitlenamewidth}{10cm}           % for the 'classic' style, if you want to force the width allocated to your name and avoid line breaks. be careful though, the length is normally calculated to avoid any overlap with your personal info; use this at your own typographical risks...

% personal data
\name{Néhémie}{Strupler}
\title{PostDoc Fellowship at the RCAC}                               
\address{Kurdela Sokak 67/10}{34435 Istanbul}{Turkey}
%\phone[mobile]{+901~(234)~567~890}                   
\phone[fixed]{+90~212~393~7600}                    
%\phone[fax]{+3~(456)~789~012}                      
\email{nehemie.strupler@etu.unistra.fr}                               
\homepage{nehemie.github.io}                         
% \extrainfo{additional information}                 
% \photo[64pt][0.4pt]{picture}   
% \quote{Some quote} 

% to show numerical labels in the bibliography (default is to show no labels); only useful if you make citations in your resume
%\makeatletter
%\renewcommand*{\bibliographyitemlabel}{\@biblabel{\arabic{enumiv}}}
%\makeatother
%\renewcommand*{\bibliographyitemlabel}{[\arabic{enumiv}]}% CONSIDER REPLACING THE ABOVE BY THIS

% bibliography with mutiple entries
%\usepackage{multibib}
%\newcites{book,misc}{{Books},{Others}}


% ------------------------------------------------------------------------------
%            content
% ------------------------------------------------------------------------------
\begin{document}
% --------       letter       --------------------------------------------------

% recipient data
\recipient{Research Center for Anatolian Civilizations}{Koç University \\
Istanbul, Turkey}
\date{December 10th, 2015}
\opening{Dear Members of the RCAC Fellowship Committee,}
\closing{Sincerely, \\ \vspace{0.5cm} \includegraphics[width=2cm]{signature.png}}
%\enclosure[Attached]{curriculum vit\ae{}}   
\makelettertitle

I am very pleased to apply for a Post-Doc Fellowship at the Research Center for
Anatolian Civilizations. I am an archaeologist specializing in the spatial
organization of urban settlement in Anatolia, especially of the Hittite
civilization. I am also particularly interested in developing theory and methods
for exploring archaeological data and making reproducible and transparent
research. I am currently finishing my dissertation on \textit{The habitation quarter
of Bogazköy-Hattusa during the Bronze Age} at the University of Strasbourg
(France) and Münster  (Germany) and will complete by May of 2016. I am also
working part-time as IT-responsible at the Istanbul Branch of the German
Archaeological Institute.

My proposed research project, \textit{Visualize Diachronic Human-environment
Interactions: A Model for the Milesian Peninsula}, examines how the
consideration of the micro-region is fundamental to generate models about human
organization within the landscape (exchange network, settlement patterns,
hinterland occupation and commidity flows). This project is based on my
involvement with the \textit{Project Panormos Survey} started in 2015.
%, where I could contribute to create the computational structure, on which this
%project develops.
First, I will focus on traditional narrative reconstructions and their
visualization based on distribution maps, by using spatial point patterns and
surface modeling. Second, I plan to develop an innovative research on
visualizing uncertainty in time and space by introducing statistical procedures
(respectively aoristic and Monte Carlo analysis). Moreover, I would like to use
this fellowship as an opportunity to advocate for Open Access and reproducible
research among my peers. 

I have visited RCAC before on many occasions with always a positive experience,
and I am convinced that being part of the international group of researchers at
RCAC will be highly motivating to achieve my own scientific goals. I am excited
to partake in or help organize, symposia related to the survey in
Anatolia. I also believe that teaching is an integral aspect of any research
program, and I look forward to working with students at Koç University on the
topic of Hittite archaeology, spatial organization, as well as developing skills on how to
map and disseminate data with open and reproducible standard. 

 
Please do not hesitate to contact me should you require any additional
materials. I would welcome the opportunity to discuss my qualifications in
greater detail. Thank you for your consideration.


\makeletterclosing

%University of Strasbourg (France) and University of Münster (Germany)



\end{document}

